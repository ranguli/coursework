\documentclass[11pt]{article}
\usepackage{amsmath,amssymb,epsfig,paralist,tabto}

\setlength{\textwidth}{7in}
\setlength{\topmargin}{-0.575in}
\setlength{\textheight}{9.25in}
\setlength{\oddsidemargin}{-.25in}
\setlength{\evensidemargin}{-.25in}

\setlength{\parindent}{0mm}

\reversemarginpar
\setlength{\marginparsep}{-15mm}

\newenvironment{tabbedenum}[1]
 {\NumTabs{#1}\inparaenum\let\latexitem\item
  \def\item{\def\item{\tab\latexitem}\latexitem}}
 {\endinparaenum}

\newcommand{\rmv}[1]{}
\newcommand{\bemph}[1]{{\bfseries\itshape#1}}
\newcommand{\N}{\mathbb{N}}
\newcommand{\Z}{\mathbb{Z}}
\newcommand{\imply}{\to}
\newcommand{\bic}{\leftrightarrow}

% Here the user must define certain strings for this homework assignment
%
\def\CourseCode{COMP 1002}		% e.g. 38
\def\AssignmentNo{1}			% e.g. 1
\def\DateHandedOut{Winter, 2017}	% e.g. September 18, 2002
\def\DateDue{January 25, 2017}		% e.g. October 1, 2002
\def\TimeDue{11:59pm}			% e.g. 11 am

\begin{document}

\noindent
\rmv{Computer Science }
\CourseCode \hfill \DateHandedOut\\

\begin{center}
Homework Assignment \#\AssignmentNo\\
Due: \DateDue, by \TimeDue\\

\textbf{Joshua Murphy} \\
\end{center}
\hrule\smallskip

\begin{tabbedenum}{2}
\item{ {\em Solution:} $(p \wedge \neg (\neg p \vee q)) \vee (p \wedge q) )$
    
    hello } 
\item{ {\em Solution: $(p \wedge \neg (\neg p \vee q)) \vee (p \wedge q) )$}} 
\end{tabbedenum}

\bigskip

\NumTabs{2}
\begin{inparaenum}
\item text
\tab\item text
\tab\item text
\tab\item text
\tab\item text
\tab\item text   
\end{inparaenum}

\end{document} 
