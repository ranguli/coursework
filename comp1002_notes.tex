\documentclass[titlepage]{article}
\usepackage[english]{babel}
\usepackage[utf8]{inputenc}
\usepackage{fancyhdr, amsfonts, amsmath}
\usepackage{glossaries}

\pagestyle{fancy}
\fancyhf{}
\rhead{Joshua Murphy}
\lhead{Course Notes - COMP 1002}
\rfoot{Page \thepage}
\title{COMP 1002 Course Notes}
\author{Joshua Murphy}
\date{\today} 

\makeglossaries

\newglossaryentry{Negation}
{
    name=Negation,
    description={The same as saying $NOT$, or using $\neg$ Symbol}
}
\longnewglossaryentry{Implication}
{
    name=Implication,
    description={Asserts that $P$ is true on the condition that $Q$ is also true. For example, it can't rain without clouds but it can be cloudy without rain.}
}

\begin{document}
\maketitle
\tableofcontents{}
\pagebreak

\section{Stuff}
\subsection{Implication}
\begin{equation}
p \to q \end{equation}
This is read as ``\textit{p implies q.}'' In this case $P$ = it is raining, 
and $Q$ = there are clouds. \gls{Implication}
Insert truth table

\subsection{Biconditional}
\begin{equation}
p \iff q \end{equation}
This is read as ``\textit{p if and only if q.}'' Both $P$ and $Q$ must be the
same value. This differs from $\wedge$ in which both values must be
\textit{true.} For example, $P$ = taking a flight and $Q$ = buying a ticket. It
is false when they have opposite values. You can take the flight if and only if
you buy a ticket.

\section{Hmm}
\subsection{Canonical CNF}

$\equiv \in \notin \emptyset \forall \exists \rightarrow \leftrightarrow \sigma
\psi \phi $
\section{Predicates, Quantifiers, Sets}

\subsection{Sets}

A set is a collection of objects:

$ \emptyset $ for the empty set \\
$ \mathbb{S} = \{1, 2, 3\}$ \\ 
$ \mathbb{N} = \{ 1,3,3\cdots\}$  is the set of \textit{natural} numbers. \\
$ \mathbb{Z} = \{\cdots, -2, -1, 0, 1, 2 \cdots\} $ is the set of integers \\

So, on forth  for real numbers, complex, rational, etc.

\subsubsection{Set Elements}

Means that element $a$ is in set S: 
\begin{equation}
 a \in S 
\end{equation}
Means $a$ is not in S:
\begin{equation}
 a \notin S
\end{equation}
Susan is not a bankteller: 
\begin{equation}
 Susan \notin Banktellers 
\end{equation}

Deductive Arguments:
    An argument is valid if the premises can’t all be true without the conclusion also being true.
    An argument is valid if the truth of all its premises forces the conclusion to be true.
    An argument is valid if it would be inconsistent for all its premises to be true and its conclusion to be false.
    An argument is valid if its conclusion follows with certainty from its premises.
    An argument is valid if it has no counterexample, that is, a possible situation that makes all the premises true and the conclusion false.

Modus Ponens: If it is a car, then it has wheels. It is a car. Therefore, it has wheels. 

Modus tollens: If it is a car, then it has wheels. It does not have wheels. Therefore, it is not a car.


\subsection{Test}
Here I talked about \gls{Negation} and 
\printglossaries
\end{document}
