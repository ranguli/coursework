\documentclass[11pt]{article}
\usepackage{amsmath,amssymb,fullpage,listings}

\setlength{\textwidth}{7in}
\setlength{\topmargin}{-0.575in}
\setlength{\textheight}{9.25in}
\setlength{\oddsidemargin}{-.25in}
\setlength{\evensidemargin}{-.25in}

\reversemarginpar
\setlength{\marginparsep}{-15mm}

\newcommand{\bemph}[1]{{\bfseries\itshape#1}}
\newcommand{\N}{\mathbb{N}}
\newcommand{\Z}{\mathbb{Z}}
\newcommand{\imply}{\to}
\newcommand{\bic}{\leftrightarrow}

% Here the user must define certain strings for this homework assignment
%
\def\CourseCode{COURSE 1001}		% e.g. B38
\def\AssignmentNo{1}			% e.g. 1
\def\DateHandedOut{Winter, 2018}	% e.g. September 18, 2002
\def\DateDue{January 25, 2018}		% e.g. October 1, 2002
\def\TimeDue{11:59pm}			% e.g. 11 am
\def\Name{John Doe}
\def\StudentNumber{2048195}

\begin{document}

\noindent
\CourseCode \hfill \DateHandedOut\\

\begin{center}
Homework Assignment \#\AssignmentNo\\
Due: \DateDue, by \TimeDue\\



\textbf{\Name  (ID: \StudentNumber) } \\

%Student Email:%
mystudentemail@uni.edu.ca
\end{center}

\hrule\smallskip


% 
\begin{enumerate} 
\item \textbf{A simple equation answer} \marginpar{}
\begin{enumerate}
\item
{\em Solution: \\ 
    \( B \leftrightarrow \neg A \wedge \neg C  \\
       C \leftrightarrow \neg B \wedge A 
        \)
            
} 
\end{enumerate}

%  
\item \textbf{Logic Truth Table} \marginpar{}
\begin{enumerate} 
\item {\em Solution: \\ 
    $(\neg P \wedge \neg Q \wedge R)
    \vee (Q \wedge R) \vee (P \wedge R) $

    \begin{displaymath}
        \begin{array}{|c|c|c|c|c|c|c|}
            P & Q & R & (\neg P \wedge \neg Q \wedge R) & (Q 
            \wedge R) & (P \wedge R) & (\neg P \wedge \neg Q
            \wedge R) \vee (Q \wedge R) \vee (P \wedge R)\\ 
            \hline 

            T & T & T & F & T & T & T  \\
            T & F & T & F & T & T & T  \\ 
            F & F & T & T & F & F & T  \\
            F & T & T & F & T & T & T  \\
            
        \end{array}
    \end{displaymath}

}
% 
\item {\em Solution: \\ 
    $ (P \wedge Q) \vee (\neg P \wedge Q) \vee (P \wedge \neg Q) \vee (\neg P
        \wedge \neg Q) $
    
    \begin{displaymath}
        \begin{array}{|c|c|c|c|c|c|}
            
            P & Q & (P \wedge Q) & (\neg P \wedge Q) & (P \wedge \neg Q) & (\neg P
            \wedge \neg Q) \\
            \hline 

            T & T & T & F & F & F    \\
            T & F & F & F & T & T    \\ 
            F & T & F & T & F & F    \\
            F & F & F & F & F & T    \\
            
        \end{array}
    \end{displaymath}

}
\end{enumerate} 


%  
\item \textbf{Equations with workings} \marginpar{}\\
\begin{enumerate} 
\item {\em Solution: \\ 

    \( \neg (P \to P \wedge Q) \wedge Q  \\
       \neg ( (\neg P \vee P) \wedge Q) \wedge Q \\
       \neg (T \wedge Q) \wedge Q \\
       F \vee \neg Q \wedge Q \\
       F \vee F \\
       F\)
}
\item {\em Solution: \\ 

    \( (T \to P) \wedge (P \to F) \\
       ( \neg T \vee P) \wedge (\neg P \vee F) \\
        ( F \vee P) \wedge (\neg P \vee F) \\ 
        P \wedge \neg P \\
        F \)

}
\end{enumerate} 
%

%  
\item \textbf{Simple one-line response or word problem} \marginpar{}\\
\begin{enumerate} 
\item {\em Solution: \\ 
    1. If Toronto, then Niagara Falls \\
    2. If not Montreal, then not Niagara Falls. \\
    3. Not Montreal and Toronto, but either or. \\
}
\end{enumerate} 
%

%  
\item \textbf{One function snippet} \marginpar{}\\
\begin{enumerate} 
\item {\em Solution: \\ 

\begin{lstlisting}
Get values for X, list L.
FPOS and IND equal -1.
While FPOS equals -1 and IND is not -1,
    if LIND equals X then,
        FPOS equals IND.
    otherwise,
        IND increases by one
return FPOS.
\end{lstlisting}
}

\end{enumerate} 
%

%  
\item \textbf{Mid-sized code snippet} \marginpar{}\\
\begin{enumerate} 
\item {\em Solution: \\ 

\begin{lstlisting}
Get list L containing 15 pairs of numbers, 
Where n is the number of elements in L.

WINS, TIES, LOSSES = 0
IND = 1

while (IND $\leq$ n) 
    If $L_{IND}$ > $L_{IND}$ + 1, increase LOSSES by one 
    If $L_{IND}$ = $L_{IND}$ + 1, increase TIES by one 
    Else, increase WINS by one 
    Increase IND by two.

Print "WINS, LOSSES, TIES"
If WINS = 15 and LOSSES and TIES both equal 0:
    Say, "Congratulations on an undefeated season."
\end{lstlisting}
}

\end{enumerate} 
%

\end{enumerate} 
\end{document}  
