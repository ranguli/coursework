\section{Propositional Logic}


\subsection{Implication}
In logic, an \gls{implication} is a 


\begin{equation}p \to q \end{equation}
This is read as ``\textit{p implies q.}'' In this case $P$ = it is raining, 
and $Q$ = there are clouds. 
Insert truth table
\subsubsection{Example:}
$p \to q$
\subsection{Biconditional}
\begin{equation}
p \iff q \end{equation}
This is read as ``\textit{p if and only if q.}'' Both $P$ and $Q$ must be the
same value. This differs from $\wedge$ in which both values must be
\textit{true.} For example, $P$ = taking a flight and $Q$ = buying a ticket. It
is false when they have opposite values. You can take the flight if and only if
you buy a ticket.

$\equiv \in \notin \emptyset \forall \exists \rightarrow \leftrightarrow \sigma
\psi \phi $

