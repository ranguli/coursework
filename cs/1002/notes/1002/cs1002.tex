\documentclass[titlepage]{article}
\usepackage[english]{babel}
\usepackage[utf8]{inputenc}
\usepackage{fancyhdr, amsfonts, amsmath}

\pagestyle{fancy}
\fancyhf{}
\rhead{Joshua Murphy}
\lhead{Course Notes - COMP 1002}
\rfoot{Page \thepage}
\title{COMP 1002 Course Notes}
\author{Joshua Murphy}
\date{\today} 
 
\begin{document}
\maketitle
\tableofcontents{}
\pagebreak

\section{Previou Section??}
\subsection{Canonical CNF}

$\equiv \in \notin \emptyset \forall \exists \rightarrow \leftrightarrow \sigma
\psi \phi $
\section{Predicates, Quantifiers, Sets}

\subsection{Sets}

A set is a collection of objects:

$ \emptyset $ for the empty set \\
$ \mathbb{S} = \{1, 2, 3\}$ \\ 
$ \mathbb{N} = \{ 1,3,3\cdots\}$  is the set of \textit{natural} numbers. \\
$ \mathbb{Z} = \{\cdots, -2, -1, 0, 1, 2 \cdots\} $ is the set of integers \\

So, on forth  for real numbers, complex, rational, etc.

\subsubsection{Set Elements}

Means that element $a$ is in set S: 
\begin{equation}
 a \in S 
\end{equation}
Means $a$ is not in S:
\begin{equation}
 a \notin S
\end{equation}
Susan is not a bankteller: 
\begin{equation}
 Susan \notin Banktellers 
\end{equation}

\subsubsection{Test}

\section{Proof by Cases}
\section{Exhaustive Proof}

\section{Recommended Textbook Questions}
\begin{enumerate}
    \item Section 1.6, Exercises 1, 3, 5, 9, 13, 15
    \item Section 1.7, Exercises 1, 3, 5
\end{enumerate}
\end{document}
